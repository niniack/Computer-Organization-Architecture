\documentclass[11pt]{exam}
\newcommand{\myname}{Nishant Aswani}
\newcommand{\mynetid}{nsa325}
\newcommand{\myemail}{nsa325@nyu.edu}
\newcommand{\myhwtype}{Homework}
\newcommand{\myhwnum}{1}
\newcommand{\mycoursenumber}{ENGR-UH 3510}
\newcommand{\myclassname}{Computer Organization and Architecture}
\newcommand{\mylecture}{0}
\newcommand{\mysection}{Z}

% Prefix for numedquestion's
\newcommand{\questiontype}{Question}

% Use this if your "written" questions are all under one section
% For example, if the homework handout has Section 5: Written Questions
% and all questions are 5.1, 5.2, 5.3, etc. set this to 5
% Use for 0 no prefix. Redefine as needed per-question.
\newcommand{\writtensection}{0}

\usepackage{amsmath, amsfonts, amsthm, amssymb}  % Some math symbols
\usepackage{enumerate}
\usepackage{enumitem}
\usepackage{graphicx}
\usepackage{hyperref}
\usepackage[all]{xy}
\usepackage{wrapfig}
\usepackage{fancyvrb}
\usepackage[T1]{fontenc}
\usepackage{listings}
\usepackage{fancyhdr}

\usepackage{centernot}
\usepackage{mathtools}
\DeclarePairedDelimiter{\ceil}{\lceil}{\rceil}
\DeclarePairedDelimiter{\floor}{\lfloor}{\rfloor}
\DeclarePairedDelimiter{\card}{\vert}{\vert}

% Uncomment the following line to get Solarized-themed source listings
% You will have had to already installed the solarized-light package
% https://github.com/jez/latex-solarized
%
%\usepackage{solarized-light}

\setlength{\parindent}{0pt}
\setlength{\parskip}{5pt plus 1pt}
\pagestyle{empty}

\def\indented#1{\list{}{}\item[]}
\let\indented=\endlist

\newcounter{questionCounter}
\newcounter{partCounter}[questionCounter]

\newenvironment{namedquestion}[1]{%
    \addtocounter{questionCounter}{1}%
    \setcounter{partCounter}{0}%
    \vspace{.2in}%
        \noindent{\bf #1}%
    \vspace{0.3em} \hrule \vspace{.1in}%
}{}

\newenvironment{numedquestion}[0]{%
	\stepcounter{questionCounter}%
    \vspace{.2in}%
        \ifx\writtensection\undefined
        \noindent{\bf \questiontype \; \arabic{questionCounter}. }%
        \else
          \if\writtensection0
          \noindent{\bf \questiontype \; \arabic{questionCounter}. }%
          \else
          \noindent{\bf \questiontype \; \writtensection.\arabic{questionCounter} }%
        \fi
    \vspace{0.3em} \hrule \vspace{.1in}%
}{}

\newenvironment{alphaparts}[0]{%
  \begin{enumerate}[label=\textbf{(\alph*)}]
}{\end{enumerate}}

\newenvironment{arabicparts}[0]{%
  \begin{enumerate}[label=\textbf{\arabic{questionCounter}.\arabic*})]
}{\end{enumerate}}

\newenvironment{questionpart}[0]{%
  \item
}{}

\newcommand{\answerbox}[1]{
\begin{framed}
\vspace{#1}
\end{framed}}

\pagestyle{head}

\headrule
\header{\textbf{NYU Abu Dhabi}}%
{\textbf{}}%
{\textbf{Division of Engineering}}

\pagestyle{head}

\begin{document}

\begin{center}
  \includegraphics[scale=0.15]{source/NYUAD-alt-logo.jpg}
\end{center}

{\vspace{1.5em}}

\begin{center}
    \Huge{\textbf{\mycoursenumber}}\\
    {\vspace{0.5em}}
    \Huge{\textbf{\myclassname}}
\end{center}

{\vspace{10em}}

\begin{center}
  \begin{tabular}{|rp{5.0cm}lll|}
    \hline
    &  &  &  & \\
    &  &  &  & \\
    \Large{\textbf{Name:}} & \Large{\myname}
    
    \  &  &  & \\
    \Large{\textbf{Net ID:}} & \Large{\mynetid}
    
    \  &  &  & \\
    \Large{\textbf{Assignment Title:}} & \Large{\myhwtype{} \myhwnum}
    
    \
    
    \  &  &  & \\
    \hline
  \end{tabular}
\end{center}

\

{\newpage}


\thispagestyle{plain}
\begin{center}
  {\Large \mycoursenumber{} \myhwtype{} \myhwnum} \\
  \myname{} (\myemail{}) \\
  \today
\end{center}

\setcounter{questionCounter}{0}

\begin{namedquestion}{Question 1.2}
Design for Moore's Law: The idea is similar to \textbf{option g} as it suggests accounting for new and upcoming technology in system design. \\

Use Abstraction to Simplify Design: \textbf{option h}, partially relying on existing sensors means that certain technology does not necessarily have to be redesigned. \\

Making the Common Case Fast: \textbf{option d}, express elevators travel to most popular floors, allowing people to skip unnecessary wait time. \\

Performance via Parallelism: \textbf{option b}, because often several suspension bridge cables share the load. It can also refer to dependability via redundancy. \\

Performance via Pipelining: \textbf{option a}, as it suggests specialization and assignment of tasks across a larger task. \\

Performance via Prediction: \textbf{option c}, because these systems carry out predictions based on not necessarily certain weather/wind values. \\

Hierarchy of Memories: \textbf{option e}, because library reserve desks hold recently requested books. Hence, procuring the books when picking up is made much faster, as it is "cached" and easy to access.\\

Dependability via Redundancy: \textbf{option f}, as speed is increased indirectly by increasing gate area, making the transistor more dependable. 

\end{namedquestion}

\begin{namedquestion}{Question 1.12.1}
Following the idea that the largest clock rate means the best performance, implies that P1 has the better performance. 

Using the classic CPU performance equation:

 \[\text{P1 CPU Time} = \frac{(5.0E9 \times 0.9)}{4E9} = 1.125s\]
 \[\text{P2 CPU Time} = \frac{(1.0E9 \times 0.75)}{3E9} = 0.25s\]
 
Hence, P2 has the better performance, however, this evaluation relies on all three factors.
\end{namedquestion}

\begin{namedquestion}{Question 1.12.2}
 \[\text{P1 CPU Time}= \frac{(1.0E9 \times 0.9)}{4E9} = 0.225s\]
 
 We see that P1 requires 0.225s seconds per program. We can now calculate the number of instructions P2 can carry out in this amount of time. 
 
 \[\text{P2 Instructions}= \frac{(3E9\times0.225)}{0.75} = 0.9E9 \text{ instructions}\]
 
Hence, P2 can carry out fewer instructions P1 in the same amount of time.
\end{namedquestion}

\begin{namedquestion}{Question 1.12.3}
MIPS is an instruction execution rate, and thus it performs inversely to execution time. Thus, a higher MIPS value means faster computation. 

\[\text{P1 MIPS}= \frac{5.0E9}{1.125\times1E6} = 4.4E3 \text{ MIPS}\]
\[\text{P2 MIPS}= \frac{1.0E9}{0.25\times1E6} = 4.0E3 \text{ MIPS}\]

According to the MIPS metric, P1 seems to have a better performance. However, this is counter to the answer found in 1.12.1, where P2 had a better CPU time. 

MIPS fails to be a great performance measure as it does not account for instruction count. Hence, variance between programs on the same machine can lead to a wide range of MIPS values.

\end{namedquestion}

\begin{namedquestion}{Question 1.12.4}

To calculate MFLOPS for both machines, we assume that 40\% of the instructions are floating point operations. We must also recalculate execution time for the FP operations.

\[\text{P1 MFLOPS}= \frac{5.0E9 \times 0.4}{\frac{(5.0E9 \times 0.4 \times 0.9)}{4E9}\times1E6} = 4.4E3 \text{ MFLOPS}\]

\[\text{P2 MFLOPS}= \frac{1.0E9 \times 0.4}{\frac{(1.0E9 \times 0.4 \times 0.75)}{3E9}\times1E6} = 4.0E3 \text{ MFLOPS}\]


\end{namedquestion}

\begin{namedquestion}{Question 1.13.1}
Reducing the time for FP operations by 20\% would result in: 

\[0.8*70s = 56s \text{ for FP operations, meaning a 14s reduction}\]

a 14s reduction to 236s for the entire program.


\end{namedquestion}
\newpage

\begin{namedquestion}{Question 1.13.3}
Branch instructions are 40s in the 250s program. 

\[\frac{40}{250} = 0.16\]

This means that the time for branch instructions makes up 16\% of the program time. It is not possible to reduce program time by 20\% by simply reducing the time for branch instructions. 


\end{namedquestion}

\end{document}

