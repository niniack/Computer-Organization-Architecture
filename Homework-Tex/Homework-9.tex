\documentclass[11pt]{exam}
\newcommand{\myname}{Nishant Aswani}
\newcommand{\mynetid}{nsa325}
\newcommand{\myemail}{nsa325@nyu.edu}
\newcommand{\myhwtype}{Homework}
%%%%%%%%%%%%%%%%%%%%%%%%%%%%%%%%
\newcommand{\myhwnum}{9}
%%%%%%%%%%%%%%%%%%%%%%%%%%%%%%%%
\newcommand{\mycoursenumber}{ENGR-UH 3511}
\newcommand{\myclassname}{Computer Organization and Architecture}

\newcommand{\cc}[1]{\texttt{#1}}

% Prefix for numedquestion's
\newcommand{\questiontype}{Question}

% Use this if your "written" questions are all under one section
% For example, if the homework handout has Section 5: Written Questions
% and all questions are 5.1, 5.2, 5.3, etc. set this to 5
% Use for 0 no prefix. Redefine as needed per-question.
\newcommand{\writtensection}{0}

\usepackage{amsmath, amsfonts, amsthm, amssymb}  % Some math symbols
\usepackage{enumerate}
\usepackage{enumitem}
\usepackage{graphicx}
\usepackage{hyperref}
\usepackage[all]{xy}
\usepackage{wrapfig}
\usepackage{fancyvrb}
\usepackage[T1]{fontenc}
\usepackage{listings}
\lstset{
  basicstyle=\ttfamily,
  mathescape
}
\usepackage{fancyhdr}
\usepackage{booktabs}
\usepackage{makecell}
\usepackage{hhline}
\usepackage[utf8]{inputenc}

\usepackage[sorting=none,style=numeric]{biblatex}
\addbibresource{refs.bib}

% \usepackage{centernot}
\usepackage{mathtools}
\DeclarePairedDelimiter{\ceil}{\lceil}{\rceil}
\DeclarePairedDelimiter{\floor}{\lfloor}{\rfloor}
\DeclarePairedDelimiter{\card}{\vert}{\vert}

% Uncomment the following line to get Solarized-themed source listings
% You will have had to already installed the solarized-light package
% https://github.com/jez/latex-solarized
%
%\usepackage{solarized-light}

\setlength{\parindent}{0pt}
\setlength{\parskip}{5pt plus 1pt}
\pagestyle{empty}

\def\indented#1{\list{}{}\item[]}
\let\indented=\endlist

\newcounter{questionCounter}
\newcounter{partCounter}[questionCounter]

\newenvironment{namedquestion}[1]{%
    \addtocounter{questionCounter}{1}%
    \setcounter{partCounter}{0}%
    \vspace{.2in}%
        \noindent{\bf #1}%
    \vspace{0.3em} \hrule \vspace{.1in}%
}{}

\newenvironment{numedquestion}[0]{%
	\stepcounter{questionCounter}%
    \vspace{.2in}%
        \ifx\writtensection\undefined
        \noindent{\bf \questiontype \; \arabic{questionCounter}. }%
        \else
          \if\writtensection0
          \noindent{\bf \questiontype \; \arabic{questionCounter}. }%
          \else
          \noindent{\bf \questiontype \; \writtensection.\arabic{questionCounter} }%
        \fi
    \vspace{0.3em} \hrule \vspace{.1in}%
}{}

\newenvironment{alphaparts}[0]{%
  \begin{enumerate}[label=\textbf{(\alph*)}]
}{\end{enumerate}}

\newenvironment{arabicparts}[0]{%
  \begin{enumerate}[label=\textbf{\arabic{questionCounter}.\arabic*})]
}{\end{enumerate}}

\newenvironment{questionpart}[0]{%
  \item
}{}

\newcommand{\answerbox}[1]{
\begin{framed}
\vspace{#1}
\end{framed}}

\pagestyle{head}

\headrule
\header{\textbf{NYU Abu Dhabi}}%
{\textbf{}}%
{\textbf{Division of Engineering}}

\pagestyle{head}

\begin{document}

\begin{center}
  \includegraphics[scale=0.15]{etc/NYUAD-alt-logo.jpg}
\end{center}

{\vspace{1.5em}}

\begin{center}
    \Huge{\textbf{\mycoursenumber}}\\
    {\vspace{0.5em}}
    \Huge{\textbf{\myclassname}}
\end{center}

{\vspace{10em}}

\begin{center}
  \begin{tabular}{|rp{5.0cm}lll|}
    \hline
    &  &  &  & \\
    &  &  &  & \\
    \Large{\textbf{Name:}} & \Large{\myname}
    
    \  &  &  & \\
    \Large{\textbf{Net ID:}} & \Large{\mynetid}
    
    \  &  &  & \\
    \Large{\textbf{Assignment Title:}} & \Large{\myhwtype{} \myhwnum}
    
    \
    
    \  &  &  & \\
    \hline
  \end{tabular}
\end{center}

\

{\newpage}


\thispagestyle{plain}
\begin{center}
  {\Large \mycoursenumber{} \myhwtype{} \myhwnum} \\
  \myname{} (\myemail{}) \\
  \today
\end{center}

\setcounter{questionCounter}{0}

\begin{namedquestion}{Problem 3.1}

Baseline Performance per loop iteration: 

Delays = 
LD(1+3) + MULTD(1+4) + DIVD(1+10) + LD(1+3) + FADD.D(1+2) + FADD.D(1+2) + FSD(1+1) + INT ADD/SUB(1 + 1 + 1) + BRANCH(1+1) = \textbf{37}\\

\end{namedquestion}

\begin{namedquestion}{Problem 3.2}

\begin{lstlisting}
Loop:   fld     f2,0(Rx) 
I0:     fmul.d  f2,f0,f2 
I1:     fdiv.d  f8,f2,f0 
I2:     fld     f4,0(Ry) 
I3:     fadd.d  f4,f0,f4 
I4:     fadd.d  f10,f8,f2 
I5:     fsd     f4,0(Ry) 
I6:     addi    Rx,Rx,8 
I7:     addi    Ry,Ry,8 
I8:     sub     x20,x4,Rx 
I9:     bnz     x20,Loop 
\end{lstlisting}

We know that tag I0 depends on tag Loop, hence there will be a stall there. We then update the code by adding \textit{3 stalls after the Loop tag}. 

We also see that tag I1 has a RAW dependency on tag I0, hence we add \textit{4 stalls after the I0 tag}. 

Tag I2 has no dependencies on any of the previous instructions. 

Tag I3 has a RAW dependency on Tag I2, hence we add \textit{2 stall cycles after Tag I2}. 

I4 has a RAW dependency on I1. the DIV.D instruction requires 10 cycles, but only 5 have been completed so far. Hence, we must add \textit{5 stall cycles before Tag I4}

I5 has a RAW dependency on I3, but the instruction has resolved by then. 

I6, I7, I8, I9 do not have any dependencies. 

\newpage

Final:

\begin{lstlisting}
Loop:   fld     f2,0(Rx) 
<stall FLD>
<stall FLD>
<stall FLD> 
I0:     fmul.d  f2,f0,f2 
<stall FMULD>
<stall FMULD>
<stall FMULD>
<stall FMULD>
I1:     fdiv.d  f8,f2,f0 
I2:     fld     f4,0(Ry) 
<stall FLD>
<stall FLD>
<stall FLD>
I3:     fadd.d  f4,f0,f4 
<stall FDIVD>
<stall FDIVD>
<stall FDIVD>
<stall FDIVD>
<stall FDIVD>
I4:     fadd.d  f10,f8,f2 
I5:     fsd     f4,0(Ry) 
I6:     addi    Rx,Rx,8 
I7:     addi    Ry,Ry,8 
I8:     sub     x20,x4,Rx 
I9:     bnz     x20,Loop 
\end{lstlisting}

Delays = 
LD(1+3) + MULD(1+4) + DIVD(1+5) + LD(1+3) + FADD.D(1) + FADD.D(1) + FSD(1) + INT ADD/SUB(1 + 1 + 1) + BRANCH(1+1) = \textbf{27}\\ 

\end{namedquestion}

\newpage

\begin{namedquestion}{Problem 3.3}

\begingroup
    \centering
    \def\arraystretch{1.5}
        \begin{tabular}{cc}
            \toprule
            Execution Pipe 1 & Execution Pipe 2\\
            \midrule
            fld & \\
            <stall> & \\
            <stall> & \\
            <stall> & \\
            fmuld & \\    
            <stall> & \\
            <stall> & \\
            <stall> & \\
            <stall> & \\
            fdiv.d & fld\\
            & <stall>\\
            & <stall>\\
            & <stall>\\
            fadd.d & \\
            <stall> & \\
            <stall> & \\
            <stall> & \\
            <stall> & \\
            <stall> & \\
            fadd.d & fsd\\
            addi & addi\\
            sub & bnz\\
            & <stall>\\

            \bottomrule
        \end{tabular}
    \label{fig:cacheTable}
    \medskip
\endgroup

In this case we only require \textbf{23 cycles}

\end{namedquestion}
\newpage

\begin{namedquestion}{Problem 3.4}

If N and N+1 are carried out in parallel and N+1 finishes earlier, this could be hazardous: 
\begin{itemize}
\item if N and N+1 were somehow storing their result to the same register. Ordering would imply that N+1 would have the latest result, but if N is slower, it will overwrite and be the latest result
\item if the pipeline were to abruptly stop and restart, then the last executed instruction would be N+1, even though N wasn't completed. 
\end{itemize}

I1 and I2 are sent in as a multiple-issue and we see that I2 completes before I1, because there are still stalls in use for I1 even after I3 has begun executing

\end{namedquestion}

\newpage 
\begin{namedquestion}{Problem 3.5}

\begingroup
    \centering
    \def\arraystretch{1.5}
        \begin{tabular}{cc}
            \toprule
            Execution Pipe 1 & Execution Pipe 2\\
            \midrule
            fld (Loop) & fld (I2)\\
            <stall> & <stall>\\
            <stall> & <stall>\\
            <stall> & <stall>\\
            fmul.d (I0) & fadd.d(I3)\\    
            <stall> & <stall>\\
            <stall> & <stall>\\
            <stall> & fsd(I5)\\
            <stall> & <stall>\\
            fdiv.d(I1) & addi(I6)\\
            <stall> & addi(I7)\\
            <stall> & sub(I8)\\
            <stall> & \\
            <stall> & \\
            <stall> & \\
            <stall> & \\
            <stall> & \\
            <stall> & \\
            <stall> & \\
            <stall> & \\
            fadd.d (I4) & bnz\\
            & <stall>\\

            \bottomrule
        \end{tabular}
    \label{fig:cacheTable}
    \medskip
\endgroup

In this case we only require \textbf{22 cycles}

\end{namedquestion}

\newpage

\begin{namedquestion}{Problem 3.6 Part A}

Given 22 total cycles, we count 15 total cycles that did not initiate any instructions. \textbf{Hence, 0.681 fraction of cycles are wasted}.

\end{namedquestion}

\begin{namedquestion}{Problem 3.6 Part B}

U1 is unroll 1; U2 is unroll 2

\begingroup
    \centering
    \def\arraystretch{1.5}
        \begin{tabular}{cc}
            \toprule
            Execution Pipe 1 & Execution Pipe 2\\
            \midrule
            fld (Loop U1) & fld (I2 U1)\\
            fld (Loop U2) & fld (I2 U2)\\
            <stall> & <stall>\\
            <stall> & <stall>\\
            fmul.d (I0 U1) & fadd.d(I3 U1)\\   
            fmul.d (I0 U2) & fadd.d(I3 U2)\\    
            <stall> & <stall>\\
            <stall> & fsd(I5 U1)\\
            <stall> & fsd(I5 U2)\\
            fdiv.d(I1 U1) & addi(I6 U1)\\
            fdiv.d(I1 U2) & addi(I6 U2)\\
            <stall> & addi(I7 U1)\\
            <stall> & addi(I7 U2)\\
            <stall> & sub (I8 U1)\\
            <stall> & sub (I8 U2)\\
            <stall> & \\
            <stall> & \\
            <stall> & \\
            <stall> & \\
            <stall> & \\
            fadd.d (I4 U1) &\\
            fadd.d (I4 U2) &\\
            bnz & \\
            & <stall>\\
            

            \bottomrule
        \end{tabular}
    \label{fig:cacheTable}
    \medskip
\endgroup

\end{namedquestion}

\begin{namedquestion}{Problem 3.6C}

After unrolling, we have 24 cycles for 2 loops, so we are doing 12 cycles per loop.

Without unrolling, we had 22 cycles per loop.

Speedup = $\frac{Old}{New}$ = $\frac{22}{12}$ = 1.83

\textbf{Unrolling makes it 1.833 times faster.}

\end{namedquestion}

\printbibliography

\end{document}

% \begingroup
%     \medskip
%     \centering
%     \def\arraystretch{1.5}
%         \begin{tabular}{cc}
%             \toprule
%             RAW hazard & stall cycles\\
%             \midrule
%             Ex to 1st & 2\\
%             Mem to 1st & 2\\
%             Ex to 2nd & 1\\
%             Mem to 2nd & 1\\
%             Ex to 1st and Mem to 2nd & 2\\
%             \bottomrule
%         \end{tabular}
%     \label{fig:c2table2}
%     \medskip
% \endgroup


